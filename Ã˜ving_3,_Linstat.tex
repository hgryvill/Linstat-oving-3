\documentclass[]{article}
\usepackage{lmodern}
\usepackage{amssymb,amsmath}
\usepackage{ifxetex,ifluatex}
\usepackage{fixltx2e} % provides \textsubscript
\ifnum 0\ifxetex 1\fi\ifluatex 1\fi=0 % if pdftex
  \usepackage[T1]{fontenc}
  \usepackage[utf8]{inputenc}
\else % if luatex or xelatex
  \ifxetex
    \usepackage{mathspec}
  \else
    \usepackage{fontspec}
  \fi
  \defaultfontfeatures{Ligatures=TeX,Scale=MatchLowercase}
\fi
% use upquote if available, for straight quotes in verbatim environments
\IfFileExists{upquote.sty}{\usepackage{upquote}}{}
% use microtype if available
\IfFileExists{microtype.sty}{%
\usepackage{microtype}
\UseMicrotypeSet[protrusion]{basicmath} % disable protrusion for tt fonts
}{}
\usepackage[margin=1in]{geometry}
\usepackage{hyperref}
\hypersetup{unicode=true,
            pdftitle={Untitled},
            pdfauthor={Astrid Langsrud, Håkon Gryvill},
            pdfborder={0 0 0},
            breaklinks=true}
\urlstyle{same}  % don't use monospace font for urls
\usepackage{color}
\usepackage{fancyvrb}
\newcommand{\VerbBar}{|}
\newcommand{\VERB}{\Verb[commandchars=\\\{\}]}
\DefineVerbatimEnvironment{Highlighting}{Verbatim}{commandchars=\\\{\}}
% Add ',fontsize=\small' for more characters per line
\usepackage{framed}
\definecolor{shadecolor}{RGB}{248,248,248}
\newenvironment{Shaded}{\begin{snugshade}}{\end{snugshade}}
\newcommand{\KeywordTok}[1]{\textcolor[rgb]{0.13,0.29,0.53}{\textbf{#1}}}
\newcommand{\DataTypeTok}[1]{\textcolor[rgb]{0.13,0.29,0.53}{#1}}
\newcommand{\DecValTok}[1]{\textcolor[rgb]{0.00,0.00,0.81}{#1}}
\newcommand{\BaseNTok}[1]{\textcolor[rgb]{0.00,0.00,0.81}{#1}}
\newcommand{\FloatTok}[1]{\textcolor[rgb]{0.00,0.00,0.81}{#1}}
\newcommand{\ConstantTok}[1]{\textcolor[rgb]{0.00,0.00,0.00}{#1}}
\newcommand{\CharTok}[1]{\textcolor[rgb]{0.31,0.60,0.02}{#1}}
\newcommand{\SpecialCharTok}[1]{\textcolor[rgb]{0.00,0.00,0.00}{#1}}
\newcommand{\StringTok}[1]{\textcolor[rgb]{0.31,0.60,0.02}{#1}}
\newcommand{\VerbatimStringTok}[1]{\textcolor[rgb]{0.31,0.60,0.02}{#1}}
\newcommand{\SpecialStringTok}[1]{\textcolor[rgb]{0.31,0.60,0.02}{#1}}
\newcommand{\ImportTok}[1]{#1}
\newcommand{\CommentTok}[1]{\textcolor[rgb]{0.56,0.35,0.01}{\textit{#1}}}
\newcommand{\DocumentationTok}[1]{\textcolor[rgb]{0.56,0.35,0.01}{\textbf{\textit{#1}}}}
\newcommand{\AnnotationTok}[1]{\textcolor[rgb]{0.56,0.35,0.01}{\textbf{\textit{#1}}}}
\newcommand{\CommentVarTok}[1]{\textcolor[rgb]{0.56,0.35,0.01}{\textbf{\textit{#1}}}}
\newcommand{\OtherTok}[1]{\textcolor[rgb]{0.56,0.35,0.01}{#1}}
\newcommand{\FunctionTok}[1]{\textcolor[rgb]{0.00,0.00,0.00}{#1}}
\newcommand{\VariableTok}[1]{\textcolor[rgb]{0.00,0.00,0.00}{#1}}
\newcommand{\ControlFlowTok}[1]{\textcolor[rgb]{0.13,0.29,0.53}{\textbf{#1}}}
\newcommand{\OperatorTok}[1]{\textcolor[rgb]{0.81,0.36,0.00}{\textbf{#1}}}
\newcommand{\BuiltInTok}[1]{#1}
\newcommand{\ExtensionTok}[1]{#1}
\newcommand{\PreprocessorTok}[1]{\textcolor[rgb]{0.56,0.35,0.01}{\textit{#1}}}
\newcommand{\AttributeTok}[1]{\textcolor[rgb]{0.77,0.63,0.00}{#1}}
\newcommand{\RegionMarkerTok}[1]{#1}
\newcommand{\InformationTok}[1]{\textcolor[rgb]{0.56,0.35,0.01}{\textbf{\textit{#1}}}}
\newcommand{\WarningTok}[1]{\textcolor[rgb]{0.56,0.35,0.01}{\textbf{\textit{#1}}}}
\newcommand{\AlertTok}[1]{\textcolor[rgb]{0.94,0.16,0.16}{#1}}
\newcommand{\ErrorTok}[1]{\textcolor[rgb]{0.64,0.00,0.00}{\textbf{#1}}}
\newcommand{\NormalTok}[1]{#1}
\usepackage{graphicx,grffile}
\makeatletter
\def\maxwidth{\ifdim\Gin@nat@width>\linewidth\linewidth\else\Gin@nat@width\fi}
\def\maxheight{\ifdim\Gin@nat@height>\textheight\textheight\else\Gin@nat@height\fi}
\makeatother
% Scale images if necessary, so that they will not overflow the page
% margins by default, and it is still possible to overwrite the defaults
% using explicit options in \includegraphics[width, height, ...]{}
\setkeys{Gin}{width=\maxwidth,height=\maxheight,keepaspectratio}
\IfFileExists{parskip.sty}{%
\usepackage{parskip}
}{% else
\setlength{\parindent}{0pt}
\setlength{\parskip}{6pt plus 2pt minus 1pt}
}
\setlength{\emergencystretch}{3em}  % prevent overfull lines
\providecommand{\tightlist}{%
  \setlength{\itemsep}{0pt}\setlength{\parskip}{0pt}}
\setcounter{secnumdepth}{0}
% Redefines (sub)paragraphs to behave more like sections
\ifx\paragraph\undefined\else
\let\oldparagraph\paragraph
\renewcommand{\paragraph}[1]{\oldparagraph{#1}\mbox{}}
\fi
\ifx\subparagraph\undefined\else
\let\oldsubparagraph\subparagraph
\renewcommand{\subparagraph}[1]{\oldsubparagraph{#1}\mbox{}}
\fi

%%% Use protect on footnotes to avoid problems with footnotes in titles
\let\rmarkdownfootnote\footnote%
\def\footnote{\protect\rmarkdownfootnote}

%%% Change title format to be more compact
\usepackage{titling}

% Create subtitle command for use in maketitle
\newcommand{\subtitle}[1]{
  \posttitle{
    \begin{center}\large#1\end{center}
    }
}

\setlength{\droptitle}{-2em}
  \title{Untitled}
  \pretitle{\vspace{\droptitle}\centering\huge}
  \posttitle{\par}
  \author{Astrid Langsrud, Håkon Gryvill}
  \preauthor{\centering\large\emph}
  \postauthor{\par}
  \predate{\centering\large\emph}
  \postdate{\par}
  \date{25 4 2018}


\begin{document}
\maketitle

\section{Introduction}\label{introduction}

Describe the problem you want to study. Why is this interesting? What
prior knowledge do you have? What do you want to achieve?

In our experiment we want to study how different factors affect our
responsiveness. The experiment was performed in the following way: A
ruler is dropped at a random point in time. The test person's
responsiveness is determined by how far down the ruler is caught. This
experiment is of interest because it gives us an idea of how mental and
physical distraction affects our responsiveness. It should be noted that
one should be careful with drawing conclusions from an experiment of
this magnitude. In order to obtain more accurate results, we should
perform many more replicates of the experiments. This would decrease the
presence of randomness in our results, and it would be clearer how each
factor affects the responsiveness.

\section{Selection of factors and
levels:}\label{selection-of-factors-and-levels}

Which factors do you think are relevant to the problem described above?
Do you expect an interaction between some of the factors? Which levels
should be used, and why do you think these are reasonable? How can you
control that the factors really are at the desired level?

We are looking at the following factors in our experiment: the gender of
the test person, physical distraction (i. e., whether the person has
been spinning around or not) and mental distraction (the person has to
talk about a given topic during the experiment). There are two main
reasons why these factors were chosen. Firstly, we thought these factors
were interesting. Secondly, these factors were expected to have some
impact on the result. On the other hand, achieving the desired level for
the different factors could be demanding. For example, spinning around
might affect the responsiveness longer than thought. It is also reason
to believe that the level of difficulty of the mental distraction was
varying; some topics could be more challenging to talk about than
others. Nevertheless, this is difficult to control. In advance, we
expected that at least physical distraction would decrease the
responsiveness. However, we did not expect that any of the factors would
have a major influence on each other. For instance, there is no reason
for us to think that spinning around affects women more than men.

\section{Selection of response
variable:}\label{selection-of-response-variable}

Which response variable will provide information about the problem
described above? Are there several response variables of interest? How
should the response be measured? What can you say about the accuracy of
these measurements?

We have chosen how far down the ruler is caught as response variable,
because this gives us a good measurement of the test subject's
responsiveness. Response time could also be used as response variable.
This is however equivalent to measuring how far the ruler has fallen, as
we can easily find the response time by applying the laws of physics.\\
The response variable was measured by taking the average of the numbers
covered by the fingers (dårlig formulering?). The response variable was
measured as the midpoint of where the hand was gripping the ruler. We
found it difficult to make accurate measurements of the response
variable, and this is regarded as an important source of error.
Considering the fact that the ruler is only 30 cm long, small
measurement errors makes relatively large errors.

\section{Choice of design:}\label{choice-of-design}

2 k factorial. Is it necessary or desirable to use a blocked design? Is
it necessary or desirable with replicates?

In order to see significant effects, two replicates of the experiment
was performed. In both replicates, all experiments were performed in
random order. \# Implementation of the experiment: Randomization.
Describe any problems with the implementation (maybe the randomization
was not followed?). Is each experiments a genuine run replicate, that is
reflects the total variability of the experiment? (Each trial should be
a performed independently and constitute a full trial.)

As we see it, there are two main problems with the performance of the
experiment. Firstly, it is almost impossible to make two genuine run
replicates; our responsiveness improves as we keep doing the
experiments. Secondly, the experiments are not completely independent,
as discussed in ``Selection of factors and levels''. On the other hand,
the fact that the order of the experiments in the replicates are
independent (and not the same, random order in both replicates)
decreases the possibility of systematic error. Performing the
experiments in the same order in both replicates could lead to the same
error being done in both experiments.

\section{Analysis of data}\label{analysis-of-data}

Calculation of effects and assessment of statistical significance. Check
the assumptions. Important: residual plots!

\begin{Shaded}
\begin{Highlighting}[]
\NormalTok{y1 =}\StringTok{ }\KeywordTok{c}\NormalTok{(}\DecValTok{16}\NormalTok{, }\DecValTok{8}\NormalTok{, }\DecValTok{15}\NormalTok{, }\DecValTok{14}\NormalTok{, }\DecValTok{15}\NormalTok{, }\DecValTok{20}\NormalTok{, }\DecValTok{11}\NormalTok{, }\DecValTok{0}\NormalTok{)}
\NormalTok{y2 =}\StringTok{ }\KeywordTok{c}\NormalTok{(}\DecValTok{13}\NormalTok{, }\DecValTok{11}\NormalTok{, }\DecValTok{14}\NormalTok{, }\DecValTok{12}\NormalTok{, }\DecValTok{20}\NormalTok{, }\DecValTok{23}\NormalTok{, }\DecValTok{14}\NormalTok{, }\DecValTok{21}\NormalTok{)}
\NormalTok{y =}\StringTok{ }\KeywordTok{c}\NormalTok{(y1, y2)}
\KeywordTok{library}\NormalTok{(FrF2)}
\end{Highlighting}
\end{Shaded}

\begin{verbatim}
## Loading required package: DoE.base
\end{verbatim}

\begin{verbatim}
## Loading required package: grid
\end{verbatim}

\begin{verbatim}
## Loading required package: conf.design
\end{verbatim}

\begin{verbatim}
## 
## Attaching package: 'DoE.base'
\end{verbatim}

\begin{verbatim}
## The following objects are masked from 'package:stats':
## 
##     aov, lm
\end{verbatim}

\begin{verbatim}
## The following object is masked from 'package:graphics':
## 
##     plot.design
\end{verbatim}

\begin{verbatim}
## The following object is masked from 'package:base':
## 
##     lengths
\end{verbatim}

\begin{Shaded}
\begin{Highlighting}[]
\NormalTok{plan <-}\StringTok{ }\KeywordTok{FrF2}\NormalTok{(}\DataTypeTok{nruns=}\DecValTok{8}\NormalTok{, }\DataTypeTok{nfactors=}\DecValTok{3}\NormalTok{, }\DataTypeTok{replications =} \DecValTok{2}\NormalTok{, }\DataTypeTok{randomize=}\OtherTok{FALSE}\NormalTok{)}
\end{Highlighting}
\end{Shaded}

\begin{verbatim}
## creating full factorial with 8 runs ...
\end{verbatim}

\begin{Shaded}
\begin{Highlighting}[]
\NormalTok{plan <-}\StringTok{ }\KeywordTok{add.response}\NormalTok{(plan,y)}
\NormalTok{plan}
\end{Highlighting}
\end{Shaded}

\begin{verbatim}
##    run.no run.no.std.rp  A  B  C Blocks  y
## 1       1           1.1 -1 -1 -1     .1 16
## 2       2           2.1  1 -1 -1     .1  8
## 3       3           3.1 -1  1 -1     .1 15
## 4       4           4.1  1  1 -1     .1 14
## 5       5           5.1 -1 -1  1     .1 15
## 6       6           6.1  1 -1  1     .1 20
## 7       7           7.1 -1  1  1     .1 11
## 8       8           8.1  1  1  1     .1  0
## 9       9           1.2 -1 -1 -1     .2 13
## 10     10           2.2  1 -1 -1     .2 11
## 11     11           3.2 -1  1 -1     .2 14
## 12     12           4.2  1  1 -1     .2 12
## 13     13           5.2 -1 -1  1     .2 20
## 14     14           6.2  1 -1  1     .2 23
## 15     15           7.2 -1  1  1     .2 14
## 16     16           8.2  1  1  1     .2 21
## class=design, type= full factorial 
## NOTE: columns run.no and run.no.std.rp  are annotation, 
##  not part of the data frame
\end{verbatim}

\begin{Shaded}
\begin{Highlighting}[]
\NormalTok{lm3 <-}\StringTok{ }\KeywordTok{lm}\NormalTok{(y}\OperatorTok{~}\NormalTok{.}\OperatorTok{^}\DecValTok{3}\NormalTok{,}\DataTypeTok{data=}\NormalTok{plan)}
\KeywordTok{summary}\NormalTok{(lm3)}
\end{Highlighting}
\end{Shaded}

\begin{verbatim}
## 
## Call:
## lm.default(formula = y ~ .^3, data = plan)
## 
## Residuals:
##      1      2      3      4      5      6      7      8      9     10 
##  1.687 -1.687 -1.687  1.688 -1.688  1.688  1.688 -1.688 -1.687  1.688 
##     11     12     13     14     15     16 
##  1.687 -1.688  1.687 -1.688 -1.688  1.688 
## 
## Coefficients:
##                Estimate Std. Error t value Pr(>|t|)
## (Intercept)      12.375      2.386   5.185    0.121
## A1               -1.875      2.386  -0.786    0.576
## B1               -2.375      2.386  -0.995    0.502
## C1               -0.875      2.386  -0.367    0.776
## Blocks.2          3.625      3.375   1.074    0.477
## A1:B1            -1.125      2.386  -0.471    0.720
## A1:C1             0.375      2.386   0.157    0.901
## A1:Blocks.2       2.625      3.375   0.778    0.579
## B1:C1            -3.625      2.386  -1.519    0.371
## B1:Blocks.2       1.625      3.375   0.481    0.714
## C1:Blocks.2       4.375      3.375   1.296    0.418
## A1:B1:C1         -1.188      1.688  -0.704    0.610
## A1:B1:Blocks.2    1.625      3.375   0.481    0.714
## A1:C1:Blocks.2    1.375      3.375   0.407    0.754
## B1:C1:Blocks.2    2.375      3.375   0.704    0.610
## 
## Residual standard error: 6.75 on 1 degrees of freedom
## Multiple R-squared:  0.9015, Adjusted R-squared:  -0.4779 
## F-statistic: 0.6535 on 14 and 1 DF,  p-value: 0.7636
\end{verbatim}

\begin{Shaded}
\begin{Highlighting}[]
\NormalTok{lm3r <-}\StringTok{ }\KeywordTok{lm}\NormalTok{(y}\OperatorTok{~}\NormalTok{(A}\OperatorTok{+}\NormalTok{B}\OperatorTok{+}\NormalTok{C)}\OperatorTok{^}\DecValTok{3}\NormalTok{,}\DataTypeTok{data=}\NormalTok{plan)}
\KeywordTok{summary}\NormalTok{(lm3r)}
\end{Highlighting}
\end{Shaded}

\begin{verbatim}
## 
## Call:
## lm.default(formula = y ~ (A + B + C)^3, data = plan)
## 
## Residuals:
##    Min     1Q Median     3Q    Max 
##  -10.5   -1.5    0.0    1.5   10.5 
## 
## Coefficients:
##             Estimate Std. Error t value Pr(>|t|)    
## (Intercept)  14.1875     1.4073  10.081 7.99e-06 ***
## A1           -0.5625     1.4073  -0.400    0.700    
## B1           -1.5625     1.4073  -1.110    0.299    
## C1            1.3125     1.4073   0.933    0.378    
## A1:B1        -0.3125     1.4073  -0.222    0.830    
## A1:C1         1.0625     1.4073   0.755    0.472    
## B1:C1        -2.4375     1.4073  -1.732    0.122    
## A1:B1:C1     -1.1875     1.4073  -0.844    0.423    
## ---
## Signif. codes:  0 '***' 0.001 '**' 0.01 '*' 0.05 '.' 0.1 ' ' 1
## 
## Residual standard error: 5.629 on 8 degrees of freedom
## Multiple R-squared:  0.4518, Adjusted R-squared:  -0.02784 
## F-statistic: 0.942 on 7 and 8 DF,  p-value: 0.5249
\end{verbatim}

\begin{Shaded}
\begin{Highlighting}[]
\KeywordTok{cubePlot}\NormalTok{(lm3,}\StringTok{"A"}\NormalTok{,}\StringTok{"B"}\NormalTok{,}\StringTok{"C"}\NormalTok{)}
\end{Highlighting}
\end{Shaded}

\includegraphics{Øving_3,_Linstat_files/figure-latex/unnamed-chunk-1-1.pdf}

\begin{Shaded}
\begin{Highlighting}[]
\KeywordTok{MEPlot}\NormalTok{(lm3) }\CommentTok{# main effects plot}
\end{Highlighting}
\end{Shaded}

\includegraphics{Øving_3,_Linstat_files/figure-latex/unnamed-chunk-1-2.pdf}

\begin{Shaded}
\begin{Highlighting}[]
\KeywordTok{IAPlot}\NormalTok{(lm3) }\CommentTok{# interaction effect plots}
\end{Highlighting}
\end{Shaded}

\includegraphics{Øving_3,_Linstat_files/figure-latex/unnamed-chunk-1-3.pdf}

\begin{Shaded}
\begin{Highlighting}[]
\NormalTok{effects <-}\StringTok{ }\DecValTok{2}\OperatorTok{*}\NormalTok{lm3}\OperatorTok{$}\NormalTok{coeff}
\NormalTok{effects}
\end{Highlighting}
\end{Shaded}

\begin{verbatim}
##    (Intercept)             A1             B1             C1       Blocks.2 
##         24.750         -3.750         -4.750         -1.750          7.250 
##          A1:B1          A1:C1    A1:Blocks.2          B1:C1    B1:Blocks.2 
##         -2.250          0.750          5.250         -7.250          3.250 
##    C1:Blocks.2       A1:B1:C1 A1:B1:Blocks.2 A1:C1:Blocks.2 B1:C1:Blocks.2 
##          8.750         -2.375          3.250          2.750          4.750
\end{verbatim}

\begin{Shaded}
\begin{Highlighting}[]
\NormalTok{n <-}\StringTok{ }\KeywordTok{length}\NormalTok{(y)}

\CommentTok{#dev.copy2pdf(file="lima3PartoR.pdf")}

\NormalTok{rres <-}\StringTok{ }\KeywordTok{rstudent}\NormalTok{(lm3)}
\KeywordTok{plot}\NormalTok{(lm3r}\OperatorTok{$}\NormalTok{fitted,rres)}
\end{Highlighting}
\end{Shaded}

\includegraphics{Øving_3,_Linstat_files/figure-latex/unnamed-chunk-1-4.pdf}

\begin{Shaded}
\begin{Highlighting}[]
\KeywordTok{library}\NormalTok{(nortest)}
\KeywordTok{library}\NormalTok{(MASS)}
\KeywordTok{qqnorm}\NormalTok{(rres)}
\KeywordTok{qqline}\NormalTok{(rres)}
\end{Highlighting}
\end{Shaded}

\includegraphics{Øving_3,_Linstat_files/figure-latex/unnamed-chunk-1-5.pdf}

\begin{Shaded}
\begin{Highlighting}[]
\CommentTok{# dev.copy2pdf(file="lima3qq.pdf")}
\KeywordTok{library}\NormalTok{(nortest)}
\KeywordTok{ad.test}\NormalTok{(}\KeywordTok{rstudent}\NormalTok{(lm3r))}
\end{Highlighting}
\end{Shaded}

\begin{verbatim}
## 
##  Anderson-Darling normality test
## 
## data:  rstudent(lm3r)
## A = 2.7339, p-value = 2.974e-07
\end{verbatim}

\section{Conclusion and
recommendations:}\label{conclusion-and-recommendations}

Which conclusions can you draw from the experiment? Interpretation of
significant effects, main and interaction plots. Remember that plots are
illustrative and very useful for demonstrations.


\end{document}
