\documentclass[]{article}
\usepackage{lmodern}
\usepackage{amssymb,amsmath}
\usepackage{ifxetex,ifluatex}
\usepackage{fixltx2e} % provides \textsubscript
\ifnum 0\ifxetex 1\fi\ifluatex 1\fi=0 % if pdftex
  \usepackage[T1]{fontenc}
  \usepackage[utf8]{inputenc}
\else % if luatex or xelatex
  \ifxetex
    \usepackage{mathspec}
  \else
    \usepackage{fontspec}
  \fi
  \defaultfontfeatures{Ligatures=TeX,Scale=MatchLowercase}
\fi
% use upquote if available, for straight quotes in verbatim environments
\IfFileExists{upquote.sty}{\usepackage{upquote}}{}
% use microtype if available
\IfFileExists{microtype.sty}{%
\usepackage{microtype}
\UseMicrotypeSet[protrusion]{basicmath} % disable protrusion for tt fonts
}{}
\usepackage[margin=1in]{geometry}
\usepackage{hyperref}
\hypersetup{unicode=true,
            pdftitle={Linear statistical models, project 3},
            pdfauthor={Astrid Langsrud, Håkon Gryvill},
            pdfborder={0 0 0},
            breaklinks=true}
\urlstyle{same}  % don't use monospace font for urls
\usepackage{graphicx,grffile}
\makeatletter
\def\maxwidth{\ifdim\Gin@nat@width>\linewidth\linewidth\else\Gin@nat@width\fi}
\def\maxheight{\ifdim\Gin@nat@height>\textheight\textheight\else\Gin@nat@height\fi}
\makeatother
% Scale images if necessary, so that they will not overflow the page
% margins by default, and it is still possible to overwrite the defaults
% using explicit options in \includegraphics[width, height, ...]{}
\setkeys{Gin}{width=\maxwidth,height=\maxheight,keepaspectratio}
\IfFileExists{parskip.sty}{%
\usepackage{parskip}
}{% else
\setlength{\parindent}{0pt}
\setlength{\parskip}{6pt plus 2pt minus 1pt}
}
\setlength{\emergencystretch}{3em}  % prevent overfull lines
\providecommand{\tightlist}{%
  \setlength{\itemsep}{0pt}\setlength{\parskip}{0pt}}
\setcounter{secnumdepth}{0}
% Redefines (sub)paragraphs to behave more like sections
\ifx\paragraph\undefined\else
\let\oldparagraph\paragraph
\renewcommand{\paragraph}[1]{\oldparagraph{#1}\mbox{}}
\fi
\ifx\subparagraph\undefined\else
\let\oldsubparagraph\subparagraph
\renewcommand{\subparagraph}[1]{\oldsubparagraph{#1}\mbox{}}
\fi

%%% Use protect on footnotes to avoid problems with footnotes in titles
\let\rmarkdownfootnote\footnote%
\def\footnote{\protect\rmarkdownfootnote}

%%% Change title format to be more compact
\usepackage{titling}

% Create subtitle command for use in maketitle
\newcommand{\subtitle}[1]{
  \posttitle{
    \begin{center}\large#1\end{center}
    }
}

\setlength{\droptitle}{-2em}
  \title{Linear statistical models, project 3}
  \pretitle{\vspace{\droptitle}\centering\huge}
  \posttitle{\par}
  \author{Astrid Langsrud, Håkon Gryvill}
  \preauthor{\centering\large\emph}
  \postauthor{\par}
  \predate{\centering\large\emph}
  \postdate{\par}
  \date{25 4 2018}


\begin{document}
\maketitle

\section{Introduction}\label{introduction}

In our experiment we want to study how different factors affect our
responsiveness. The experiment was performed in the following way: A
ruler is dropped at a random point in time. The test person's
responsiveness is determined by how far down the ruler is caught. This
experiment is of interest because it gives us an idea of how mental and
physical distraction affects our responsiveness. It should be noted that
one should be careful with drawing conclusions from an experiment of
this magnitude. In order to obtain more accurate results, we should
perform many more replicates of the experiments. This would decrease the
presence of randomness in our results, and it would be clearer how each
factor affects the responsiveness.

\section{Selection of factors and
levels:}\label{selection-of-factors-and-levels}

We are looking at the following factors in our experiment: the gender of
the test person, physical distraction (i. e., whether the person has
been spinning around or not) and mental distraction (the person has to
talk about a given topic during the experiment). There are two main
reasons why these factors were chosen. Firstly, we thought these factors
were interesting. Secondly, these factors were expected to have some
impact on the result. On the other hand, achieving the desired level for
the different factors could be demanding. For example, spinning around
might affect the responsiveness longer than thought. It is also reason
to believe that the level of difficulty of the mental distraction was
varying; some topics could be more challenging to talk about than
others. Nevertheless, this is difficult to control. In advance, we
expected that at least physical distraction would decrease the
responsiveness. However, we did not expect that any of the factors would
have a major influence on each other. For instance, there is no reason
for us to think that spinning around affects women more than men.

\section{Selection of response
variable:}\label{selection-of-response-variable}

We have chosen how far down the ruler is caught as response variable,
because this gives us a good measurement of the test subject's
responsiveness. Response time could also be used as response variable.
This is however equivalent to measuring how far the ruler has fallen, as
we can easily find the response time by applying the laws of physics.\\
The response variable was measured by taking the average of the numbers
covered by the fingers (dårlig formulering?). The response variable was
measured as the midpoint of where the hand was gripping the ruler. We
found it difficult to make accurate measurements of the response
variable, and this is regarded as an important source of error.
Considering the fact that the ruler is only 30 cm long, small
measurement errors makes relatively large errors.

\section{Choice of design:}\label{choice-of-design}

In order to see significant effects, two replicates of the experiment
was performed. In both replicates, all experiments were performed in
random order.

\section{Implementation of the
experiment:}\label{implementation-of-the-experiment}

As we see it, there are two main problems with the performance of the
experiment. Firstly, it is almost impossible to make two genuine run
replicates; our responsiveness improves as we keep doing the
experiments. Secondly, the experiments are not completely independent,
as discussed in ``Selection of factors and levels''. On the other hand,
the fact that the order of the experiments in the replicates are
independent (and not the same, random order in both replicates)
decreases the possibility of systematic error. Performing the
experiments in the same order in both replicates could lead to the same
error being done in both experiments.

\section{Analysis of data}\label{analysis-of-data}

Now we are going to analyse the results of performing the experiments.

The following is the summary of the model fit for two replications when
all interactions are included.

\begin{verbatim}
## 
## Call:
## lm.default(formula = y ~ (A + B + C)^3, data = plan)
## 
## Residuals:
##    Min     1Q Median     3Q    Max 
##   -6.5   -2.5    0.0    2.5    6.5 
## 
## Coefficients:
##             Estimate Std. Error t value Pr(>|t|)    
## (Intercept)  16.8125     1.3095  12.839 1.28e-06 ***
## A1            1.3125     1.3095   1.002    0.346    
## B1            0.8125     1.3095   0.620    0.552    
## C1            2.1875     1.3095   1.670    0.133    
## A1:B1         0.8125     1.3095   0.620    0.552    
## A1:C1         0.6875     1.3095   0.525    0.614    
## B1:C1         1.1875     1.3095   0.907    0.391    
## A1:B1:C1      1.6875     1.3095   1.289    0.234    
## ---
## Signif. codes:  0 '***' 0.001 '**' 0.01 '*' 0.05 '.' 0.1 ' ' 1
## 
## Residual standard error: 5.238 on 8 degrees of freedom
## Multiple R-squared:  0.4779, Adjusted R-squared:  0.02111 
## F-statistic: 1.046 on 7 and 8 DF,  p-value: 0.4696
\end{verbatim}

The summary shows that the all the p-values,
Pr(\textgreater{}\textbar{}t\textbar{}), for all the factors are larger
than 0.1, implying that none of the factors have significant impact on
the response. Also the standard error is relatively high.

However one observe that the value for factor C, distraction, is higher
than for the other factors, and the p-value is lower. Thus, it seems
like this facor might have a higher impact on the result than the other.
For the interaction factors it seems like the interaction between facor
B and C, and the interactins between all the three factors are the most
sifnificant. But also for thses the p-value is too high to conclude
anything.

Thw following plots are the main effect plot and the interaction plot
\includegraphics{Øving_3,_Linstat_files/figure-latex/unnamed-chunk-3-1.pdf}

\includegraphics{Øving_3,_Linstat_files/figure-latex/unnamed-chunk-4-1.pdf}
These plots illustrates the same trends as those given by the summary.
Factor C gives the steepest curve in the main effect plot. However, when
looking at the y-axis one sees that the line is not actually very steep.
In the interaction plot the lines for the interaction factor of B and C
are least parallell.

The next figures are the residual plot and the Q-Q-plot, testing
normality of the residuals.
\includegraphics{Øving_3,_Linstat_files/figure-latex/unnamed-chunk-5-1.pdf}
\includegraphics{Øving_3,_Linstat_files/figure-latex/unnamed-chunk-5-2.pdf}

The residuals do not seem to follow any specific pattern in the residual
plot. In the normal Q-Q-plot the residuals deviates some from the line
in the ends. In both plots it would be easier to conclude anything about
the normality of the residualt if there had been more points.

Since the two replicates was performed at two different times, it could
be interesting to see if there are any significant differences between
the two replicates. Thus, we now try to set this as a block factor, by
having the replicates in different blocks. The following is a summary of
the model where replicate number is set to an additional facort.

\begin{verbatim}
## 
## Call:
## lm.default(formula = y ~ (a + b + c)^3 + d)
## 
## Residuals:
##    Min     1Q Median     3Q    Max 
## -7.312 -2.094  0.000  2.094  7.312 
## 
## Coefficients:
##             Estimate Std. Error t value Pr(>|t|)    
## (Intercept)  16.8125     1.3658  12.309 5.36e-06 ***
## a             1.3125     1.3658   0.961    0.369    
## b             0.8125     1.3658   0.595    0.571    
## c             2.1875     1.3658   1.602    0.153    
## d            -0.8125     1.3658  -0.595    0.571    
## a:b           0.8125     1.3658   0.595    0.571    
## a:c           0.6875     1.3658   0.503    0.630    
## b:c           1.1875     1.3658   0.869    0.413    
## a:b:c         1.6875     1.3658   1.236    0.256    
## ---
## Signif. codes:  0 '***' 0.001 '**' 0.01 '*' 0.05 '.' 0.1 ' ' 1
## 
## Residual standard error: 5.463 on 7 degrees of freedom
## Multiple R-squared:  0.503,  Adjusted R-squared:  -0.0649 
## F-statistic: 0.8857 on 8 and 7 DF,  p-value: 0.5703
\end{verbatim}

This did not improve the model, as the p-values are still large for all
the factors.

\section{Conclusion and
recommendations:}\label{conclusion-and-recommendations}

Which conclusions can you draw from the experiment? Interpretation of
significant effects, main and interaction plots. Remember that plots are
illustrative and very useful for demonstrations.


\end{document}
